% originally by Sangchul Hahn, and revised by H. Choi

\newcommand{\abstractEN}[1]{\normalsize #1 \par}
\newcommand{\abstractKR}[1]{\normalsize #1 \par}

\fontsize{14}{18} \selectfont Abstract \par
\addcontentsline{toc}{chapter}{Abstract}

\bigskip
\abstractEN{
    Defect prediction studies have been actively conducted over the past decades.
    However, existing defect prediction models face challenges such as the lack of actionable messages and the cold-start problem.
    To overcome these issues, various approaches, i.e. cross-project defect prediction, unsupervised defect prediction, and the just-in-time defect prediction, have been studied.
    However, these approaches are still limited in several aspects.
    We propose a novel approach to predict defects in the change level by searching for similar changes within the existing software repositories.
    The model identifies a change as ``buggy'' if the change is similar to the existing bug-inducing change; otherwise, it predicts a change as ``clean''.
    The model then suggests a corresponding patch as an actionable message.
    Our approach can provide a new direction to the filed while addressing two main issues faced by traditional defect-prediction models.
}
\bigskip
\bigskip
\abstractKR{
    결함 예측 연구는 지난 수십 년 동안 활발하게 수행되어 왔다.
    하지만 기존의 결함 예측 모델은 실행 가능한 메시지 부족 및 콜드 스타트 문제들과 같은 어려움에 직면해 있다.
    이러한 문제점을 극복하기 위해 교차 프로젝트 결함 예측, 비지도 결함 예측, 즉각 결함 예측 등 다양한 방법들이 연구되어 왔다.
    하지만 이러한 연구들 또한 여러 측면의 한계점을 갖고 있다.
    따라서, 우리는 이를 해결하기 위해 기존의 없던 유사 수정 기반의 새로운 버그 예측 모델을 제안한다.
    이 모델은 예측 대상인 수정이 기존 소프트웨어 저장소에 있는 버그 유발 수정과 유사하다면 버그라 예측하고 유사하지 않으면 클린으로 예측한다.
    버그라고 예측된 수정에는 연관된 유사한 패치를 추천하여 실행 가능한 메시지를 제공한다.
    본 논문에서 제안하는 모델은 콜드 스타트 및 클래스 불균형 문제를 완화하고 실행 가능한 메시지에 대한 패치를 제안함으로써 결함 예측 모델의 세 가지 문제를 해결할 수 있는 가능성을 보여준다.
}

\clearpage
