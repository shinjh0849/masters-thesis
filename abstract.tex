% originally by Sangchul Hahn, and revised by H. Choi

\newcommand{\abstractEN}[1]{\normalsize #1 \par}
\newcommand{\abstractKR}[1]{\normalsize #1 \par}

\fontsize{14}{18} \selectfont Abstract \par
\addcontentsline{toc}{chapter}{Abstract}

\bigskip
\abstractEN{
    Defect prediction studies have been actively conducted over the past decades.
    However, existing defect prediction models face challenges in the lack of actionable messages, the cold-start problem, and the class imbalance problem.
    To overcome these issues, various approaches such as Cross-Project Defect Prediction (CPDP), unsupervised defect prediction, Just-in-time Defect Prediction (JIT-DP), and feature selection techniques have been studied.
    However, these studies still have limitations in several aspects.
    Hence, we propose a novel approach of predicting defects in the commit level by searching for similar commits in existing software repositories.
    The model predicts a commit as “buggy” if the commit is similar to the existing Bug Inducing Commit (BIC) and if it is not similar, it will predict it as “clean”.
    Then, it suggests a corresponding patch (bug-fixing change) to the similar BIC as an actionable message for the bug-prone change.
    Our approach shows the potential to address the three issues of defect prediction models by alleviating the cold-start and class imbalance problems and suggesting patches for actionable messages.
}
\bigskip
\bigskip
\abstractKR{
    결함 예측 연구는 지난 수십 년 동안 활발하게 수행되어 왔다.
    그러나 기존의 결함 예측 모델은 실행 가능한 메시지 부족, 콜드 스타트 문제 및 클래스 불균형 문제에서 어려움을 겪고 있다.
    이러한 문제점을 극복하기 위해 교차 프로젝트 결함 예측(CPDP), 비지도 결함 예측(Unsupervised defect prediction), JIT-DP(Just-in-time Defect Prediction), 특징 선택 기법 등 다양한 방법들이 연구되어 왔다.
    하지만 이러한 연구들 또한 여러 측면의 한계점을 갖고 있다.
    따라서, 우리는 이를 해결하기 위해 기존의 없던 유사 커밋 기반의 새로운 버그 예측 모델을 제안한다.
    이 모델은 예측 대상인 커밋ㅇ이 기존 소프트웨어 저장소에 있는 버그 유발 커밋(BIC)과 유사하다면 버그라 예측하고 유사하지 않으면 클린으로 예측한다.
    버그라고 예측된 커밋에게는 연관된 유사한 패치를 추천하여 실행 가능한 메시지를 제공한다.
    본 논문에서 제안하는 모델은 콜드 스타트 및 클래스 불균형 문제를 완화하고 실행 가능한 메시지에 대한 패치를 제안함으로써 결함 예측 모델의 세 가지 문제를 해결할 수 있는 가능성을 보여준다.
}

\clearpage
