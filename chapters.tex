% originally by Sangchul Hahn, and revised by H. Choi
\mainmatter
% Chapter
\chapter{Introduction}
\thispagestyle{fancy}
\bigskip
Along with the rise in software complexity, the cost of quality assurance and software development continues to rise too.
To address this issue, numerous studies have been conducted for automated quality assurance tasks to reduce the cost of software development and maintenance, such as automatic program repair (APR) \cite{kim2013automatic, long2016automatic, mechtaev2016angelix}, automated test case generation \cite{ali2009systematic, anand2013orchestrated, lei2008ipog} defect detection \cite{pradel2012leveraging, pradel2018deepbugs, wang2016bugram}, software defect prediction (SDP) \cite{nam2017heterogeneous, wang2016automatically, zimmermann2009cross}.
Among them, (SDP) models have been actively studied to efficiently allocate testing resources to reduce development cost.
Software defect prediction uses static code metrics as features on machine learning predictors to identify module of code as buggy or clean. 
However, like other techniques, defect prediction models face numerous challenges in practical usage.

\section{Main Issues of Defect Prediction}
There are several issues in SDP such as identifying relationship between feature metrics and the label, no consistency in adopting performance evaluation metrics, difficulty in gathering buggy data, lack of standard or general framework, lack of validity in the economic benefit, and the validity of how to retrieve buggy data \cite{arora2015open, herbold2019issues}.
Apart from these issues, we want to tackle some other important issues that challenges the research field of defect prediction.

\subsection{Actionable Messages}
One of major limitation that SDP models face is that the predicted results lack actionable or explainable messages for the developers to act upon \cite{lewis2013does}.
The traditional defect prediction model predicts source code modules, usually in a file or a method level, as risky according to their degree of complexity.
However, the prediction does not specify which part of the code is buggy or explains what problems the module is causing.
Due to this nature, it is difficult for the developer to act upon the prediction result or understand how the module is causing the program.
To alleviate this issue, just-in-time defect prediction  (JIT-DP) got into attention \cite{kamei2012large}.
JIT-DP predicts bugs in the code change level.
With finer granularity of prediction, researchers aim to provide `practical' defect prediction models for developers.
However, the finer granularity doesn't solve the problem, because if the commit gets too big, the same problem occurs.
Besides, the approach still does not explain how the commit is causing problems to the software.
In other words, the models do not provide actionable messages for the bug-prone changes.

\subsection{Cold-start Problem}
The second major problem of traditional defect prediction models is the cold-start problem.
A cold-start problem occurs when the target system lacks historical data \cite{schein2002methods}. 
Because traditional defect prediction models are built with previous versions of the target system, it is impossible to apply on projects that are just being started.
To alleviate this problem, the study of cross-project defect prediction (CPDP) and heterogeneous defect prediction (HDP) got popular.
CPDP enables defect prediction to be applied on newly starting projects because the prediction models are trained from other existing projects (cross-project).
HDP enables CPDP even when the source project and the test project have different metrics as features.
Machine learning techniques can only be applied when the training data and the test data have the same dimension of features (homogeneous), however, this case is not always met.
So, the development of HDP increases the range of projects that could be used as training data.
However, CPDP and HDP do not fully alleviate the cold-start problem in the JIT settings because the metrics need a certain degree of historical data, i.e. the time spent after the last commit (AGE), the number of developers that contributed to the commit (NDEV), the number of unique changes in the commit (NUC) in \cite{kamei2012large}.

\subsection{Contribution}
Herein, we propose a novel and potential SDP paradigm to resolve these issues.
Essentially, we search for changes similar to the target change from software repositories and then use their distance value to identify the target change as buggy or clean.
We name this model the \underline{sim}ilar commit change \underline{fin}der ({\simfin}), which searches for similar changes.
We also develop a \underline{\simfin}-based defect-prediction \underline{mo}del ({\simfinmo}) which alleviates lack of actionable messages by providing patch suggestions.
The intuition of {\simfin} is as follows.
If the target change is very similar to our searched change (with the lowest distance value, i.e. the closest change existing in a repository), it is most likely to be a buggy change.
If the target change is very different from our searched change, then it is more likely to be a clean change.
Our approach is very novel because, in the prediction phase, we do not apply any machine learning algorithm as opposed to other existing defect prediction models.
Also, we do not need a certain period of historical data to collect metrics for test data.
For prediction, {\simfinmo} simply looks if the distance ratio of the searched buggy and clean changes to the target change is higher than the cutoff value or not.
With this approach, we could alleviate the two aforementioned limitations of SDP models.

\vspace{3mm}
The contributions of our work is that:
\begin{enumerate}
    \item We propose a completely new paradigm of defect prediction approach.
    \item With our model, we can alleviate the lack of actionable messages by suggesting patches that are used to fix the searched similar change.
    \item We mitigate the cold-start problem because our model is a fully universal, and needs no single previous commit for a test instance because we do not use historical metrics.
\end{enumerate}

% Chapter 
\clearpage
\chapter{Related Work}
\thispagestyle{fancy}
\bigskip
\section{Related Work}
\subsection{Similar Commit Search}
The key part of our approach is that SimFin finds which commits in the existing repositories that are simmilar to the target commit. Existing studies that are related to this technique are code clone detection/search, code search engines, commit clustering.

\subsubsection{Code Clone Detection/Search}
Jiang et al. \cite{jiang2007deckard} proposed an approach named DECARD which represents code blocks as subtrees and uses similarity algorithms on tree data structures.
Lee et al. \cite{lee2010instant} proposed a method that uses multi-dimennsional indexing technique and kNN (k-Nearest Neighborhood) algorithm to reduce the search time while maitaining the functianlity of finding sematically similar code fragments.
They used 54 MLOC of code to make this code clone detection module.
Keivanloo et al. \cite{keivanloo2011internet} did a similar research, but their main diference is in that they used hash tables and binary search algorithm in implementing a multi-level indexing technique.
They experimented and evaluated on 266 MLOC of code bases.
White et al. \cite{white2016deep} exploited deep learning techniques that are used in natural language processing (e.g. Recursive Neural Network, or Recurrent Neural Network) to extract syntactical patterns and detect code clones with similar patterns. 

The difference between code clone detection/search and similar commit search is that they have different structures of code bases.
Commit shows how code is changed from one code to the other which contains information such as which nodes are added, deleted, updated, or moved, or the metadata of the commit such as which developer is responsible for the change, time of commit, number of changed files, and so on.
Due to their structural differences, the necessity of studying a different approach is evident. 

\subsubsection{Code Search Engine}
Bajracharya et al. \cite{bajracharya2006sourcerer} proposed a tool named Sourcerer which searches for code fragments.
The tool divides target code fragments respect to the code usage to improve the search rate.
The have divided the categories into implementation, uses, and structures.
McMillan et al. \cite{mcmillan2011exemplar} proposed Exemplar (\underline{Exe}cutable ex\underline{mpl}es \underline{ar}chive) helps find code fragments that functions as the natural language query input.
This study focused on improving the search rate by reducing the gap between the high abstraction of natural language query and low level language of source code.
Kim et al. \cite{kim2010towards} proposed a tool that when a user searches for a API document, it returns code snippets that can be helpful for the API's usage together with the document.
Kim et al. \cite{kim2018facoy} proposed FACOY (\underline{F}ind \underline{A} \underline{C}ode \underline{O}ther than \underline{Y}ours) searches for a code fragment that is similar to the user's input but not in a syntactical or semantical way but with a similar function.
Gu et al. \cite{gu2018deep} proposed CODEnn (\underline{Co}de-\underline{D}escription \underline{E}mbedding \underline{n}eural \underline{n}etwork) to find a semantical significance of the natural language query and the target code snippet.
They do this buy mapping both natural language and code snippet in high dimennsional vector space and trains a deep learning model to map these instances as close to a space if they have semantically similar.

The difference between code search and commit search is that static code and commit has different structure, just as code clone search.
Second, code search techniques are more focused on handling natural language query as input.
While some do handle code fragments, they are limited in their ability to handle longer code fragments. 
To do a fully commit search, we must be able to handle longer code bases as input to search for commits because commits can be very long.

\subsubsection{Commit Clustering}
Kreutzer et al. \cite{kreutzer2016automatic} did a study about clustering similar commits respect to their major functions (e.g. bug fixing, refactoring, etc.).
To do that, they have extracted commits that are in existing software repositories such as Git and applied LCS (Longest Common Subsequence) algorithm to retrieve a matrix of commit's similarities.
With this matrix, they applied two kinds of clustering algorithms to categorize commits that have similar scores.
Dias et al. \cite{dias2015untangling} did a similar work but with a different scope.
They categorized different changes within a commit respect to different intentions.
They studied this because with a single commit, developers change several files that are sometimes nothing to do with their intentions (i.e. tangled change).
To do this, they used IDE activity history, and applied different machine learning algorithms (i.e. binary logistic regression, random forest, naive bayes, etc.) for classification and applied hierarchical clustering to cluster them.

The difference between commit clustering and commit search is that in \cite{kreutzer2016automatic}, the clusterings are too big to find the syntactical or semantical similarities of each commits. And as for \cite{dias2015untangling}, the granularity of change is within a single commit, making it hard to scale up to search similar commits in other projects.

\subsection{Defect Prediction}
In this section, we survey the various defect prediction methods and explain how they are different from each other and from our work.
\subsubsection{Traditional Defect}
Traditional defect prediction predicts a module in different granularity as buggy or clean.
In traditional defect prediction scenario, the granularity is usually in the file-level or the method level.
They use previous version of their own project to predict the current or latter version of the project.
Munson et al. \cite{munson1992detection} built a classification model to classify if a module has high risk or not with the accuracy of 92\%.
Chidamber and Kemerer \cite{chidamber1994metrics} proposed a suite of object-oriented related metrics that could be applied in defect prediction.
Nagappan and Ball \cite{nagappan2005use} proposed code churn metrics to predict defect density of the system.
This was the first process related metrics and more process related metrics were proposed after.

\subsubsection{Cross-Project Defect Prediction}
Cross-project defect prediction (CPDP) was proposed to alleviate the cold-start problem of the traditional defect prediction because traditional defect prediction relied on previous versions of the target project.
For project with little or no previous data, it is very hard or impossible to apply defect prediction.
So CPDP uses data from other projects to learn the prediction model.
Watanabe et al. \cite{watanabe2008adapting} proposed the first CPDP approach to apply prediction model that are already built for other projects.
Ma et al. \cite{ma2012transfer} proposed Transfer Naive Bayes (TNB) that weights source instance similar to the target instances
Nam et al. \cite{nam2013transfer} proposed TCA+ to alleviate feature differencing problem in applying CPDP.

\subsubsection{Heterogeneous Defect Prediction}
Heterogeneous defect prediction was first proposed by \cite{nam2017heterogeneous}. It is a cross-project defect prediction where the source project and the target project have different feature space.
This method enables source project to have different set of features which was an impossible thing to do.
With this technique, it expanded the range of projects to be selected as training set, which is very important because collecting buggy data is very hard.
Li et al. \cite{li2018cost} proposed cost-sensitive transfer kernel canonical correlation analysis (CTKCCA) to evaluate nonlinear correlation relationship of the different features.
Li et al. \cite{li2019heterogeneous} proposed a two-staged ensemblme learning (TSEL) approach for HDP, which contains ensemble multi-kernel domain adaptation stage and ensemble data sampling stage. These stages handles seprates nonlinear correlation of the features and the imbalance class of the labels.
Tong et al. \cite{tong2019kernel} proposed a kernel spectral embedding transfer ensemble (KSETE) which addresses the class imbalance problem, finds the latent common feature space by combining kernel spectral embedding.

\subsubsection{Just-in-Time Defect Prediction}
Just-in-time defect prediction (JIT DP) tackles another problem in the traditional defect prediction.
The actionability of traditional defect prediction is limited because usually a predicted module is too big,
making it very hard for the developers to act upon to fix the bug.
In JIT DP, the granularity of the prediction is at the change-level, usually smaller than a whole source file, making it easier for the developers to act upon due to the smaller code base.
Mockus et al. \cite{mockus2000identifying} proposed the first to identify changes with respect to their specific reasons of causes: adding new features, correcting faults, and restructuring code for future changes.
Kim et al. \cite{kim2008classifying} is the first study that did a machine learning modelling for predicting buggy change of a project.
Kamei et al. \cite{kamei2016studying} empirically evaluated JIT prediction model in the context of cross-project scenario.
They found that the models improve performance when selecting models that use other similar projects, using a larger pool of dataset, and using several projects for ensemble learning.  

These various defect prediction models use machine learning for prediction.
On the contrary, our method of defect prediction does not use any machine learning algorithms for prediciton.
Eventhough we use autoencoder and kNN, it is for searching similar commits.
The predictions are made comparing the distance and cutting them with a threshold value.

% Survey some CPDP and JIT DPs. Explain what they use for their defect prediction.
% Explain how the existing DPs are different from ours. 1. We do not use ML algos in prediction. (we use auto-encoder and knn, but they don't play a role in prediction phase).

% Survey studies that finds similar commits. 

% code search engines,
% code clone studies,
% clustering of code commits,
% untangling fine-grained code changes.

% explain how their approach is different from our SimFin finding simmilar changes.

% Chapter 
\clearpage
\chapter{Preliminary Studies}
\thispagestyle{fancy}
\bigskip
This chapter explains the preliminary studies that were conducted to find different change representations.
Deciding how to represent changes is very important because the information that can be carried in a representation is very different concerning their data forms.

\section{Count-based Change Vector}
Count-based change vector representation of change has a fixed size of the sparse vector.
Each index represents a change type of an AST node.
If a certain AST node is added, the index that corresponds to the kind of change is incremented.
We capture the changed AST nodes in 4 change types, i.e. addition, deletion, updates, moves, by using a fine-grained source code differencing algorithm, Gumtree \cite{falleri2014fine}.
Because the change is represented in a numeric vector, it is suitable to be applied on numerous distance-calculating equations.
Specifically, we used the Jaccard index to find similar changes \cite{cha2007comprehensive}.

Fig. \ref{fig:cbcv_3} shows an example of two count-based change vectors that have the Jaccard index of 0.92.
Fig. \ref{fig:cbcv_1} and \ref{fig:cbcv_2} show an example of the corresponding changes.  
As depicted in the figures, we can see that both changes are similar in terms of their AST structures.
The underlined values 146 1 and 159 1 in Fig. \ref{fig:cbcv_3} means that both changes have one importDeclaration (146) and MethodInvocation (159) node added.

However, there are limitations to this representation.
Because the changes are composed of changed AST node counts, it is not able to capture the order of the changed nodes.
So changes that had very similar vectors also resulted in totally different looking changes because of different occurrences of nodes.
So to capture a more accurate representation of change, we needed a different design. 

\section{String Comparison}
To capture the order information of changes, we experimented on using the change code as a string value and used string comparing algorithms in finding similar changes.

First, we used the traditional line differencing algorithm used in git.
We didn't use Gumtree this time because Gumtree only considered the changed node without its context.
By retrieving the whole line, we could represent the change with order and context.

To reduce noise information, we only retrieved changes that both deleted lines (buggy) and added lines (patch) within the same method scope.
This way we can be sure that the change is a bug-fixing change. 
Other changes that only have either added or deleted lines are not considered.


\begin{figure}[!ht]
    \renewcommand{\arraystretch}{1}
    \centering
    \includegraphics[width=\linewidth]{figures/cbcv-1.pdf}\hfill
    \caption{A change instance in project isis.}
    \label{fig:cbcv_1}
\end{figure}
    
\begin{figure}[!ht]
    \renewcommand{\arraystretch}{1}
    \centering
    \includegraphics[width=\linewidth]{figures/cbcv-2.pdf}\hfill
    \caption{A change instance in project ignite.}
    \label{fig:cbcv_2}
\end{figure}

\begin{figure}[!ht]
    \renewcommand{\arraystretch}{1}
    \centering
    \includegraphics[width=\linewidth]{figures/cbcv-3.pdf}\hfill
    \caption{Count-based change vector if Fig. \ref{fig:cbcv_1} and \ref{fig:cbcv_2}}
    \label{fig:cbcv_3}
\end{figure}


Comparing string values take much more time than applying them to numeric values. 
So we decided to apply the SimHash algorithm to reduce the dimension and skipped comparing if they were over a certain Hamming distance \cite{uddin2011effectiveness}.
This method significantly reduced the time of similarity comparison.
We also used a mix of Jaccard index and Levenshtein distance to capture both ordering and the collective occurrence of string values. 

Fig. \ref{fig:string_1}-\ref{fig:string_4} shows an example that had a Jaccard index of 0.86. 
As depicted in the figures, changes that had the highest similarity values were usually very short. 
Some changes were longer, but they had a low similarity value.
This result seemed natural because when string values are longer there are more chances that they will be different.
So we decided on using a different method which is the {\simfin} to find similar commits in the software repositories.

For the count-based change vector and string comparison, we should find hand-designed features such as the number of AST node occurrences or string similarities.
Also, the features are limited to syntactic characteristics of code changes.
Therefore, we propose a method to find similar commits by applying deep learning so that the computer can directly derive semantic features.

\begin{figure}[!ht]
    \renewcommand{\arraystretch}{1}
    \centering
    \includegraphics[width=0.8\linewidth]{figures/string_1.pdf}\hfill
    \caption{A change instance in project lucene-solr (deleted line)}
    \label{fig:string_1}
\end{figure}

\begin{figure}[!ht]
    \renewcommand{\arraystretch}{1}
    \centering
    \includegraphics[width=0.8\linewidth]{figures/string_2.pdf}\hfill
    \caption{A change in project lucene-solr (added line)}
    \label{fig:string_2}
\end{figure}

\begin{figure}[!ht]
    \renewcommand{\arraystretch}{1}
    \centering
    \includegraphics[width=0.8\linewidth]{figures/string_3.pdf}\hfill
    \caption{A change instance in project oozie (deleted line)}
    \label{fig:string_3}
\end{figure}

\begin{figure}[!ht]
    \renewcommand{\arraystretch}{1}
    \centering
    \includegraphics[width=0.8\linewidth]{figures/string_4.pdf}\hfill
    \caption{A change instance in project oozie (added line)}
    \label{fig:string_4}
\end{figure}

% Chapter 
\clearpage
\chapter{Approach}
\thispagestyle{fancy}
\bigskip
% The figure for showing the overall approaches.
\begin{figure*}[!tbp]
\renewcommand{\arraystretch}{1}
    \centering
    \includegraphics[width=\textwidth]{figures/approach_fig.pdf}\hfill
    \caption{Overall structure of the {\simfin} and {\simfinmo} approach.}
    \label{fig:approach}
\end{figure*}
    
     

This chaper details the implementation of {\simfinmo}.
The overall approach of {\simfinmo} is illustrated in Fig. \ref{fig:approach}.

\section{Data Collection}
We mined bug-inducing commits (BICs) and clean commits from software repositories to train the {\simfin}.
First, we used the SZZ algorithm \cite{sliwerski2005changes} to collect BICs from software repositories. 
The SZZ algorithm first mined bug fixing changes (BFC) from issue tracking system such as JIRA.
The issue tracking system manages all issues that have occurred during the development of the project.
The issue is labeled according to the type, status and resolution.
To find the bug-fix issue, we look for issues that are labeled as the following: type that are labeled ``Bug'', status that are labeled ``Closed'' or ``Resolved'' and resolution that are labeled ``Fixed''.

Projects that are managed by JIRA include issue keys which are unique numbers of issues and used in commit messages.
Therefore, by using the issue key of bug fix issue, we can find BFCs.
The deleted lines in the BFC are considered to be the buggy code while the added lines are the corresponding patches.
We compare previous commit of the BFC and the actual BFC in each source, to extract deleted or replaced lines using git diff which is based on Myers diff algorithm \cite{myers1986ano}.
Then, we apply git blame command to each modified line since git blame shows the information of last commit id, the author, the timestamp and the line number of code for the line.
By using these piece of information, we mined the BIC.

After collecting the BICs, we also collect all the other changes and label it as clean changes.
We do this because {\simfin} exploits both BIC and clean changes in forming the search engine.

\section{Vector Embedding}
We vectorized the change data after mining them from software repositories
First, we collect the source code of a change applied the Gumtree~\cite{falleri2014fine} algorithm.
We use Gumtree for differencing the two source code because finer granularity captures a more precise change.
By capturing a finer granularity of change, we can accurately represent change while reducing the memory as well.
We also want to capture less false positive changes (code parts that are not actually changed but identified as changed) to compactly represent the changes.
Lastly, because the changes are represented in AST vectors, we can capture the syntactical change in the vectors so that the {\simfin} can capture the relations of the node's syntax change.

If we apply Gumtree to the code before and after a change, the changed nodes will be represented as insertion, deletion, or update of a node or a move of a sub-tree.
We only regard insertion and deletion of a node because updates and moves were mostly refactoring changes.
However, they could be some other meaningful changes.
So it can be regarded as sacrificing some instances for a denoising effect.
After we collect each remaining changes, we encode each node-change with a unique integer value. 
Then, we append each value in the order of its occurrence from top to bottom, left to right in the source code.
By doing this, we can capture the syntactic characteristic as well as the order information of the change.
After that, we append what we call the context vector.
The context vector is the a list of neighboring nodes of each node-change.
We collect the context vectors by taking the all descendant nodes of the parent node of each changed node.
Then we disregard the descendant nodes by collecting nodes that are within 3 lines of each corresponding changed node.
We also disregard duplication as there could be a number of redundant context nodes.
We have chosen to use the context vector to capture a richer information of changes, so when we search for similar changes, we not only look at the change themselves but also the context where the change have taken.
The label data for {\simfin} is also constructed in this phase.
For the label, we use the key value of each change, which is the commit id and the source file path of each change. 
The reason for making the label in such way is because when {\simfin} is given with a target commit, it identifies the closest commit in software repositories.
The key value should contain the id of the commit and the source file path of the change to return the most similar commit.

\section{{\simfin}: Similar Change Finder}
{\simfin} is composed of a deep auto encoder-decoder (AED) and a k-Nearest Neighborhood (kNN) model.
The deep AED model first encodes the embedded syntactic vectors from Gumtree and learns the semantics of changes.
Then the kNN model computes distances of each data points of changes and returns the closest \emph{k} changes and its distance.

\subsection{Number of {\simfin}s}
First, we have built one {\simfin} where all the BICs and clean changes are trained together.
This is reasonable as we are going to use the ratio of the distance value of closest BIC and the clean change to the target change in identifying the defect.
By plotting all the changes in the same encoding space, the degree of distance between the target change and both clean and buggy changes will have the same degree.

Second, we also built two separate {\simfin} where one pair of AED and kNN model is trained with buggy change instances and the other with clean change instances.
When we build two separate {\simfin} models, the encoding space of buggy and clean changes become different, hence using there distance ration can be insensible.
However, due to the data size of clean changes and buggy changes being extremely imbalanced (around 26 : 1), looking for the closest buggy and clean changes within a certain \emph{k} can show a very skewed result.

To alleviate these issues raised from imbalanced class, we trained a separate AED model to get the distance of the closest \emph{k} change instances in each buggy and clean data pool.
With \emph{k} numbers of closest distances in both pools, we hope to get the representative distance value of each pools.

For validity of choosing whether to train one or two different {\simfin}s, we compared the performance measure of the two methods in section VI.

\subsection{Auto Encoder-Decoder Model}
AED model in {\simfin} is used to learn and encode the relationship of the syntactic feature and its semantics.
First, we apply zero-padding to all the training instances to match the dimension size.
Then, the encoder encodes the vector by passing through the deep layers of the encoder network.
Then it is reconstructed by passing through the deep layers of the decoder network.
The reconstruction error is used to backpropagate through the network and update the weights to reduce the error.
The settings we used for the AED model is 5 layers for each networks, 500 nodes for each layer, 3 epochs, and a batch size of 256.
We used ReLU at each layers for the activation function and a Sigmoid function at the last layer of the decoder.
Binary cross-entropy was used for the loss function and Ada-delta was used for the optimizer.
The decoder of the network is used to update the weight of the encoder model, but we only use the encoder part of the network to encode the test projects in the prediction phase.
The BIC instances and the clean instances are trained into separate AED models.
We trained the networks separately because we made a presumption that BIC and clean changes have different characteristics.

\subsection{k-Nearest Neighborhood}
After we encode the syntactic and semantic representations of changes, we feed them to a kNN model to find similar changes which is the closest data point in the vector space.
The kNN model originally makes prediction of an instance's class with respect to the distance in the vector space.
The labels are usually a binary or multi-class of labels, however, we use commit key as the label which is a unique label.
This is done because we want to search the closest commit to the target commit.
All the label in kNN model, Fig. \ref{fig:approach}, is depicted as different icons to show that each labels are unique.
Because of this nature, it is not able to, or not sensible, to get the evaluation score of the kNN.
Similar to AED model, the kNN models are also trained separately from BIC instances and clean instances.
Thus, we finally have buggy {\simfin} and clean {\simfin} respectively.

\section{Prediction}
In the prediction phase, the target commit is vectorized and fed into the {\simfin}.
The change vectors, together with the context vectors, are generated from applying the Gumtree algorithm.
The semantic representation of the change is inferred from the encoder that is trained before-hand.
Lastly, the learned feature representation is plotted in the vector space of kNN model, which then searches for nearest changes.
Because we have different set of models, one (buggy \simfin) made from BIC and another (clean \simfin) from clean instances, we plot the target change in both vector spaces.
After plotting the target change into both spaces, we search for the closest change in each space.
Then, we can get the closest distance value in the BIC space and it divided by the closest distance value in the clean space.
By using these two values, we can compute the distance ratio as follows:
\begin{equation}
    DR = \frac{\delta_b}{\delta_c}
\end{equation}
where $\delta_b$ is a the closest distance value from buggy \simfin while $\delta_c$ is a the closest distance value from clean \simfin.
If a target change has a closer distance of a similar change from buggy \simfin than that from clean \simfin, $DR$ is always less than 1. Otherwise, $DR$ is 1 or greater.
The intuition behind this method is that if a target change is very close to the closest BIC and is very far away from the closest clean instance, it is more likely to be buggy.
On the contrary, if the target change is far away from the closest BIC instance but it is closer to the closest clean instance, it is more likely to be clean.
Here, we need to set the cutoff value for $DR$ values to decide whether the target change is buggy or clean.
If we set a cutoff as 1 for \simfinmo and $DR$ is less than this predefined cutoff value, we predict it as buggy.
Since $DR=1$ implies the target change has the same closest distance values from both boggy and clean \simfin, we use 1 as a default cutoff for \simfinmo.

After predicting a target change as buggy, then \simfinmo suggests the BFC of the closest buggy change from buggy \simfin.
This BFC can be used as a bug fix hint of the target change and can be an actionable message for a developer.

% Chapter 
\clearpage
\chapter{Experimental Setup}
\thispagestyle{fancy}
\bigskip
This chapter explains the settings of the experiment that is conducted in this study.

\section{Research Questions}
We considered two research questions to evaluate the defect prediction of {\simfinmo}.

\begin{itemize}
    \item RQ1: Was the defect prediction of {\simfinmo} potentially comparable to those of various machine learners?
    \item RQ2: How actionable are patches suggested by the {\simfinmo}?
    \item RQ3: What were the impacts of various {\simfinmo} cutoffs on prediction performance?
    \item RQ4: How effective is using the divided {\simfin} in terms of predictive performance?
\end{itemize}

We investigated the effectiveness of {\simfinmo} by comparing its prediction performance with the existing baseline.
Furthermore, we studied various aspects of {\simfinmo} using different cut-off values.

\section{Dataset}
We used 193 active, Java projects in the Apache Software Foundation (ASF) to construct {\simfin}.
We chose the projects that maintain active GitHub- or Jira-issued tracking systems.
A total of 110K BIC instances and 2.6M clean instances were collected for the training data to build the buggy and clean {\simfin} models.

Table \ref{tab:test_project} shows the details of the test project.
The test set used are also from ASF.
These test projects were selected by considering various buggy ratios and the different number of commits.
JIT-DP conducted at the change level remains very challenging to achieve high prediction performance.
One of the reasons is that the number of buggy commits is significantly smaller than that of clean commits.
These ratios are affected by the total number of commits in a project.
Thus, for our test data, we randomly chose six projects by considering various buggy ratios and the different number of commits.
As explained in the approach section, the two {\simfin} models are trained by the BIC instances or clean instances.

% test set list
\begin{table}[!htp]
\caption{The list of project used as a test set}
\centering
% \resizebox{\linewidth}{!}{%
\begin{tabular}{|l|l|l|l|}
\hline
Name & \# of Buggy & \# of Clean & Total \\ \hline
maven & 988 (9.2\%) & 10786 (90.8\%) & 11774 \\ \hline
ranger & 709 (12.2\%) & 5810 (87.8\%) & 6519 \\ \hline
sentry & 265 (10.8\%) & 2446 (89.2\%) & 2711 \\ \hline
sqoop & 91 (2.2\%) & 4204 (97.8\%) & 4295 \\ \hline
syncope & 1254 (4.6\%) & 26415 (95.4\%) & 27669 \\ \hline
tez & 1091 (16.5\%) & 6629 (83.5\%) & 7720 \\ \hline
median & 709 (9.2\%) & 6629 (90.8\%) &  7720\\ \hline
\end{tabular}%
% }
\newline
\label{tab:test_project}
\end{table}

\section{Baseline}
The baseline we use to compare the prediction performance is a typical JIT-DP metric reported in Kamei et al.~\cite{kamei2012large}.
The metric types are classified into five dimensions: diffusion, size, purpose, history,and experience. 

The \textbf{diffusion dimension} of a change shows how distributed a change is.
A distributed change can be measured by counting the different components of source files.
There are four features in this category: number of modified subsystems (NS), number of modified directories (ND), number of modified files (NF), and distribution of modified code(Entropy).
For our experiment, we considered the number of the subsystem as one because they are all Java projects.
For illustration, if a commit changed three files: java/src/clami/main.java, java/src/clami/utils.java, and the java/src/remi/input.java.
NS is one (java/src/), ND is two (clami/ and city/) and NF is three (main.java, input.java, and utils.java).
Entropy counts the distribution of modified lines in a source file.
\textbf{Size dimension} is the number of changed lines in a source file.
The three features of these categories are lines of code added (LA), lines of code deleted (LD), and lines of code in a file before the change (LT).
The \textbf{purpose dimension} has one feature that which is FIX.
FIX is a binary feature that labels if a commit is a BFC or not.
This is used as a feature because a BFC is more likely to introduce new bugs.
The \textbf{history dimension} is about the revision history of changes from the past to the present.
NDEV is the number of unique developers who have modified a source file.
AGE is the time between the current source and the most recent modification.
NUC is the number of unique changes in a commit.
For example, there are four source files in a commit that are A, B, C, and D. File A and B had been modified at $\alpha$ commit, file C had been modified at $\beta$ commit and file D had been modified at $\gamma$ commit.
In this case, NUC is three ($\alpha$, $\beta$, and $\gamma$). 
The \textbf{experience dimension} is the information of developers in the project.
Developers who frequently participate in the project is less likely to cause bugs because the developer understands the project well.
The experience dimension has two factors: Developer experience(EXP) and Recent developer experience(REXP).
EXP is the total number of commits the developer has created.
REXP is the number of commits that have been weighted according to the year the developer participated in.
A developer, for example, created one commit in 2017, three commits in 2018, and two commits in 2020.
REXP in 2020 is 3.25 (i.e., $\frac{2}{1} + \frac{3}{3}  + \frac{1}{4}$), and REXP in 2021 is 1.95 (i.e., $\frac{2}{2} + \frac{3}{4}  + \frac{1}{5}$).

With these metrics, an online change classification model is built as reported from Tan et al. \cite{tan2015online}.
Online change classification is proposed to address the limitation of cross-validation on change classification.
Firstly, cross-validation is not suitable in change classification because it will use future data for predicting bugs from the past.
This scenario does not match the real-world scenario because buggy change and a fixing change has a chronological relationship.
Secondly, there is a chance of mislabeling the changes. 
When we predict a buggy change in a certain period, the buggy change is labeled as buggy because there is a bug-fixing change in the future.
However, in the time the fixing didn't happen yet so it should be labeled as clean.

To address these issues, Tan et al. \cite{tan2015online} proposed an online change classification method.
They left a gap between the training data and the test data.
This gap should be an estimate of bug-fixing time so that there is enough time for the buggy changes in the trainset to be discovered.
Also, the method is to construct an online model, meaning that the training data is accumulated and updated as time goes by.
By doing this, the model is applied with multiple runs alleviating the sensitivity that can be dependent on a particular time.

In online change classification, the first two datasets cannot be used as a test set because when predicting the first dataset, the prediction time is before any bug-fixing is done and therefore there will be no buggy changes to predict from. The second data set cannot be used as well because at least one period has to skip to leave a gap.
For a fair comparison, the evaluation for {\simfinmo} does not predict the first two sets of data.

We have assessed six of the most used machine learning algorithms in defect prediction.
The algorithms we trained are BayesNet (BN), k-Nearest Neighbor (IBk), Logistic model tree (LMT), Naive Bayes (NB), and random forest (RF).

\subsection{Evaluation Metrics}
The evaluation metrics for comparing {\simfinmo} and baseline are precision, recall, F1 score and MCC.
We used a variety of evaluation metrics to assess a sound experiment and show various aspects of the predictors.
A Confusion matrix is needed to evaluate two models.
There are four metrics as shown in table \ref{tab:Confusion}.
True positive (TP) is when the actual label is true and the model predicts it as true.
False positive (FP) is when the actual label is false but the model predicts as true.
False negative (FN) is when the actual label is true but the model predicts as false.
True negative (TN) is when the actual label is false and the model predicts it as false.
Precision is the value of positive predictive that is the correct percentage of bugs among predicted bugs ($Precision = \frac{TP}{TP + FP}$) and recall is hit rate that is the percentage of predicted bugs among actual bugs ($Recall = \frac{TP}{TP + FN}$).
F1-score is the harmonic mean of the precision and recall ($F1-score = 2 \times \frac{Precision \times Recall}{Precision + Recall}$). 
MCC is the Matthews correlation coefficient that is a measure used in unbalanced labels ($MCC = \frac{TP \times TN - FP \times FN}{\sqrt{(TP + FP)(TP + FN)(TN + FP)(TN + FN)}}$. 

\begin{table}[htbp]
\caption{Confusion matrix}
\centering
% \resizebox{\linewidth}{!}{%
\begin{tabular}{|c|c|c|}\hline

\backslashbox{Predicted}{Actual} & Buggy & Clean\\ \hline
Buggy & True Positive & False Positive \\ \hline
Clean & False Negative & True Negative \\ \hline
\end{tabular}%
% }
\newline
\label{tab:Confusion}
\end{table}

% Chapter 
\clearpage
\chapter{Experiment Results}
\thispagestyle{fancy}
\bigskip
This chapter shows the experiment results of the reproduced baseline and our approach of {\simfinmo}.
Table \ref{tab:precision}- \ref{tab:mcc} shows the overall results of the baseline and our approach.
The bold values in the table indicates the highest score from the projects.


\section{RQ1: {\simfinmo} vs Baseline}
From Table \ref{tab:precision}, we can see that all most all of the baseline results are better in precision.
However, Table \ref{tab:recall} shows that {\simfinmo} always has better performance in recall.
Due to precision and recall having trade-offs with each other, it is better to see the F-measure which is the harmonic mean of precision and recall.
From Table \ref{tab:f1score}, we can see that out of 6 projects, {\simfinmo} outperforms 4 of the projects in F1 score. 
The average F1 score of {\simfinmo} is also the highest out of all the baseline machine learning algorithms.
Table \ref{tab:mcc} shows a mixed result, but overall, random forest shows the best performance in terms of MCC with the highest average.
From the results, we can state that {\simfinmo} outperforms the baseline overall.

% Precision
\begin{table}[!ht]
\caption{Precision value of each baseline and \simfinmo}
\centering
\begin{tabular}{|c|c|c|c|c|c|c|c|}
\hline
Name & BN & IBk & J48 & LMT & NB & RF & SFM \\ \hline
maven & 0.220 & 0.141 & 0.253 & 0.297 & 0.147 & \textbf{0.362} & 0.146 \\ \hline
ranger & 0.200 & 0.144 & 0.162 & 0.141 & 0.182 & \textbf{0.350} & 0.183 \\ \hline
sentry & 0.154 & 0.065 & 0.168 & 0.160 & 0.098 & \textbf{0.233} & 0.141 \\ \hline
sqoop & 0.220 & 0.153 & 0.205 & 0.243 & 0.170 & \textbf{0.379} & 0.036 \\ \hline
syncope & 0.268 & 0.099 & 0.209 & 0.200 & 0.076 & \textbf{0.280} & 0.056 \\ \hline
tez & 0.161 & 0.173 & 0.232 & 0.159 & 0.150 & \textbf{0.265} & 0.241 \\ \hline
average & 0.204 & 0.129 & 0.205 & 0.200 & 0.137 & \textbf{0.311} & 0.134 \\ \hline
\end{tabular}%

\label{tab:precision}
\end{table}

% Recall
\begin{table}[!ht]
\caption{Recall value of each baseline and \simfinmo.}
\centering
\begin{tabular}{|c|c|c|c|c|c|c|c|}
\hline
Name & BN & IBk & J48 & LMT & NB & RF & SFM \\ \hline
maven & 0.114 & 0.190 & 0.126 & 0.103 & 0.322 & 0.065 & \textbf{0.834} \\ \hline
ranger & 0.539 & 0.162 & 0.267 & 0.278 & 0.831 & 0.125 & \textbf{0.886} \\ \hline
sentry & 0.294 & 0.120 & 0.160 & 0.184 & 0.341 & 0.104 & \textbf{0.887} \\ \hline
sqoop & 0.207 & 0.193 & 0.213 & 0.261 & 0.441 & 0.107 & \textbf{0.846} \\ \hline
syncope & 0.023 & 0.078 & 0.040 & 0.009 & 0.172 & 0.019 & \textbf{0.877} \\ \hline
tez & 0.298 & 0.355 & 0.371 & 0.528 & 0.665 & 0.302 & \textbf{0.841} \\ \hline
average & 0.246 & 0.183 & 0.196 & 0.227 & 0.462 & 0.120 & \textbf{0.862} \\ \hline
\end{tabular}%
\label{tab:recall}
\end{table}


% F-1 Score
\begin{table}[!ht]
\caption{F1 score of each basline and \simfinmo}
\centering
\begin{tabular}{|c|c|c|c|c|c|c|c|}
\hline
Name & BN & IBk & J48 & LMT & NB & RF & SFM \\ \hline
maven & 0.150 & 0.162 & 0.168 & 0.153 & 0.202 & 0.111 & \textbf{0.248} \\ \hline
ranger & 0.292 & 0.152 & 0.202 & 0.187 & 0.299 & 0.184 & \textbf{0.303} \\ \hline
sentry & 0.202 & 0.085 & 0.164 & 0.171 & 0.153 & 0.143 & \textbf{0.243} \\ \hline
sqoop & 0.213 & 0.171 & 0.209 & \textbf{0.251} & 0.245 & 0.167 & 0.068 \\ \hline
syncope & 0.042 & 0.087 & 0.068 & 0.017 & \textbf{0.106} & 0.036 & 0.105 \\ \hline
tez & 0.209 & 0.232 & 0.285 & 0.244 & 0.244 & 0.282 & \textbf{0.375} \\ \hline
average & 0.185 & 0.148 & 0.183 & 0.170 & 0.208 & 0.154 & \textbf{0.224} \\ \hline
\end{tabular}%
\label{tab:f1score}
\end{table}


% MCC
\begin{table}[!ht]
\caption{MCC of each baseline and \simfinmo}
\centering
\begin{tabular}{|c|c|c|c|c|c|c|c|}
\hline
Name& BN & IBk & J48 & LMT & NB & RF & SFM \\ \hline
maven & 0.096 & 0.058 & 0.119 & \textbf{0.126} & 0.089 & 0.120 & 0.056 \\ \hline
ranger & 0.150 & 0.015 & 0.047 & 0.017 & \textbf{0.182} & 0.143 & 0.095 \\ \hline
sentry & 0.098 & -0.043 & 0.080 & 0.079 & 0.013 & 0.100 & \textbf{0.101} \\ \hline
sqoop & 0.084 & 0.011 & 0.071 & 0.119 & 0.053 & \textbf{0.137} & 0.047 \\ \hline
syncope & 0.059 & 0.021 & \textbf{0.061} & 0.027 & -0.001 & 0.056 & 0.051 \\ \hline
tez & 0.051 & 0.076 & 0.155 & 0.074 & 0.068 & \textbf{0.167} & 0.033 \\ \hline
average & 0.089 & 0.023 & 0.089 & 0.074 & 0.067 & \textbf{0.121} & 0.064 \\ \hline
\end{tabular}%
\label{tab:mcc}
\end{table}


\section{RQ2: How actionable is {\simfinmo}?}
To show how much the patches that are suggested by {\simfinmo} are helpful to developers, we manually inspected the patches and labeled each of them.
We only inspected patches that were predicted to be a true positive.
We also only considered patches were the ground truth patches are short. 
The reason we only considered short patches because most of the longer patches composed refactoring and architecture changes.
With these kind of patches, it is difficult analyze the patches correctly, due to the variety of number and domain of the used projects.
The labels our are as follows:
\begin{enumerate}
    \item \textbf{Direct hint}: the ground truth patch and the suggested patch are same changes respect to their AST node type.
    \item \textbf{Indirect hint}: the ground truth patch and the suggested patch are not the same changes respect to their AST node type, however, they have similar node types with high relevance.
    \item \textbf{No hint}: the ground truth patch and the suggested patch are does not have the same changes respect to their AST node type and has no relevance at all.
\end{enumerate}
The labeling can be very subjective as to say that changes have relevance or not.
So we include some examples to show how the manual labels are done.

In Fig. \ref{fig:direct_hint}, there are three changes: the target BIC change, ground truth patch of the target BIC change, and the suggested patch. 
By looking at the the ground truth patch, we can see that the intialization of a string variable contatains bug as it is replaced with another value (true -> false).
While they have different values, we can see that the suggested patch also changes a public static final String value.
In cases where the changed AST node types are the same between ground truth patch and the suggested patch, we label them as \emph{Direct hint}.

Similar example is shown in Fig. \ref{fig:indirect_hint}.
The introduced bug has to do with ommitting a parameter in a method implementation.
The suggeseted patch also shows that it added a method paramenter.
However, it is happening in a method invocation line of code.
And the type of the changed parameter is most likely to be a different type.
Eventhough the change contains different types of AST node, we can still infer that omission of a parameter can help fix the bug.
In these kind of cases, we label them as \emph{Indirect hint}.

Through this method of labeling, we inspected all the test projects.
Table \ref{tab:rq_2} shows the statistics of the suggested patch labeling.
We can see that 28\% ~ 60\% of the patches are shows direct or indirect hint.
The percentage of the hint respect to the suggested patch can be shown as low.
However, it is encouraging tho see this result because no other method provided an actionable message this direct in other defect prediction methods.
And to consider that the result only shows the top 1 suggestions, we can hope that the hint rate will be higher when we consider more top k instances.

\begin{figure*}[!tbp]
\renewcommand{\arraystretch}{1}
    \centering
    \includegraphics[width=\textwidth]{figures/direct_hint.png}\hfill
    \caption{An example of a direct hint in labeling suggested patches.}
    \label{fig:direct_hint}
\end{figure*}

\begin{figure*}[!tbp]
\renewcommand{\arraystretch}{1}
    \centering
    \includegraphics[width=\textwidth]{figures/indirect_hint.png}\hfill
    \caption{An example of a indirect hint in labeling suggested patches.}
    \label{fig:indirect_hint}
\end{figure*}

% RQ2
\begin{table}[!htp]
\centering
\caption{A table that shows the statistics of suggested patches and their labeling.}
\begin{tabular}{|l|l|l|l|}
\hline
{Name} & {Direct hint} & {Indirect hint} & {No hint} \\ \hline
{maven} & 6 (18\%) & 8 (23\%) & 20 (59\%) \\ \hline
{ranger} & 11 (16\%) & 16 (24\%) & 40 (60\%) \\ \hline
{sentry} & 4 (14\%) & 7 (24\%) & 18 (62\%) \\ \hline
{sqoop} & 3 (30\%) & 3 (30\%) & 4 (40\%) \\ \hline
{syncope} & 9 (12\%) & 14 (19\%) & 52 (69\%) \\ \hline
{tez} & 10 (14\%) & 10 (14\%) & 70 (72\%) \\ \hline
\end{tabular}%
\label{tab:rq_2}
\end{table}


\section{RQ3: {\simfinmo} with different cut-off values}
To provide a better concept of how well the model {\simfinmo} predicts defective modules, we have investigated the different performance values with different cut-off values.
The results are tabulated in Table \ref{tab:rq_3}.
Due to the limitation of space in the report, we have only tabulated one of the test projects, maven.
From the table, we can see that precision score is highest when the cut-off value is low (closer to zero).
However, the recall value is the lowest.
The precision score peaks when the cut-off ranges from 0.000001 to 0.1 for other projects as well.
On the other hand, precision drops pretty low when the cut-of value goes over 1.
However, recall starts to go up rapidly as the cut-off value gets higher.
The ascending of the recall is much higher than the descending of the precision yielding a good value of f1-score. 

% RQ3.
\begin{table}[!ht]
\centering
\caption{This table shows different performance metrics using different cut-off values.
The following result is from the project maven. }
\begin{tabular}{|c|c|c|c|c|}
\hline
Cut-off & \multicolumn{1}{c|}{Precision} & \multicolumn{1}{c|}{Recall} & \multicolumn{1}{c|}{F1 Score} & \multicolumn{1}{c|}{MCC} \\ \hline
0.1 & \textbf{0.286} & 0.002 & 0.004 & 0.014 \\ \hline
0.2 & 0.167 & 0.003 & 0.006 & 0.005 \\ \hline
0.3 & 0.183 & 0.036 & 0.061 & 0.023 \\ \hline
0.4 & 0.190 & 0.146 & 0.165 & 0.054 \\ \hline
0.5 & 0.196 & 0.351 & 0.252 & 0.100 \\ \hline
0.6 & 0.194 & 0.531 & \textbf{0.284} & \textbf{0.131} \\ \hline
0.7 & 0.177 & 0.650 & 0.278 & 0.120 \\ \hline
0.8 & 0.155 & 0.727 & 0.255 & 0.073 \\ \hline
0.9 & 0.149 & 0.797 & 0.250 & 0.062 \\ \hline
1 & 0.146 & 0.837 & 0.248 & 0.056 \\ \hline
2 & 0.144 & 0.924 & 0.249 & 0.063 \\ \hline
3 & 0.144 & 0.934 & 0.250 & 0.066 \\ \hline
4 & 0.144 & 0.936 & 0.250 & 0.066 \\ \hline
5 & 0.144 & 0.939 & 0.250 & 0.068 \\ \hline
6 & 0.144 & 0.940 & 0.250 & 0.067 \\ \hline
7 & 0.143 & 0.940 & 0.249 & 0.064 \\ \hline
8 & 0.143 & 0.940 & 0.248 & 0.062 \\ \hline
9 & 0.143 & 0.942 & 0.249 & 0.064 \\ \hline
10 & 0.143 & \textbf{0.942} & 0.249 & 0.064 \\ \hline
\end{tabular}%
\label{tab:rq_3}
\end{table}


\section{RQ4: Combined {\simfin} vs. Divided {\simfin}}
In section IV, we have talked about how we decided on choosing one combind design of {\simfin} or buggy/clean divided design of {\simfin}.
To show how different design of {\simfin} contribute to the prediction performance of {\simfinmo}, we compare their result.
In table \ref{tab:rq_4}, we can see that the divided {\simfin} outperforms the combined {\simfin} in precision, recall, and f1-score of every test projects.
It also outperforms 4 out of 6 test projects in MCC.
Through this result, we think that dividing the encoding space of buggy and clean changes helps retrieving the representative distance from the target change.

% RQ4.
\begin{table}[!ht]
\centering
\caption{This table shows prediction performance of combined {\simfin} and two divided {\simfin}}
\resizebox{\textwidth}{!}{%
\begin{tabular}{|l|l|l|l|l|l|l|l|l|}
\hline
\multirow{2}{*}{} & \multicolumn{4}{c|}{combined {\simfin}} & \multicolumn{4}{c|}{divided {\simfin}} \\ \cline{2-9} 
Name & Precision & Recall & F1-score & MCC & Precision & Recall & F1-Score & MCC \\ \hline
maven & 0.109 & 0.660 & 0.187 & \textbf{0.093} & \textbf{0.146} & \textbf{0.834} & \textbf{0.248} & 0.056 \\ \hline
ranger & 0.123 & 0.726 & 0.210 & 0.060 & \textbf{0.183} & \textbf{0.886} & \textbf{0.303} & \textbf{0.095} \\ \hline
sentry & 0.115 & 0.709 & 0.198 & 0.071 & \textbf{0.141} & \textbf{0.887} & \textbf{0.243} & \textbf{0.101} \\ \hline
sqoop & 0.029 & 0.495 & 0.055 & 0.042 & \textbf{0.036} & \textbf{0.846} & \textbf{0.068} & \textbf{0.047} \\ \hline
syncope & 0.054 & 0.684 & 0.100 & 0.048 & \textbf{0.056} & \textbf{0.877} & \textbf{0.105} & \textbf{0.051} \\ \hline
tez & 0.161 & 0.570 & 0.252 & \textbf{0.058} & \textbf{0.241} & \textbf{0.841} & \textbf{0.375} & 0.033 \\ \hline
\end{tabular}%
}
\label{tab:rq_4}
\end{table}

% \section{Analysis}
% Table \ref{tab:precision}- \ref{tab:mcc} shows the prediction performances between baseline and {\simfinmo} in various measures such as precision, recall, f1-score, and MCC.

% We use Friedman and Nemenyi test to statistically evaluate the performance of algorithms of {\simfinmo} and baseline. Friedman test is a non-parametric test to determine the statistical significant of the data that is classification algorithm, and usually comparing three or more data. In this paper, Friedman test is used to compare the statistical significance of evaluation metrics of all the classifiers of each project. The outputs of Friedman test are degree of freedom that is the maximum number of logically independent values, Friedman chi-squared that if the value is large, there is a relationship and if it is small, there isn't relationship and p-value that mean the relationship is statistically significant when the value is less than 0.05.
% The p-value of precision is  0.008772, and Friedman chi-squared is 17.143. The p-value of recall is 5.88E-06, and Friedman chi-squared is 34.303. Lastly, p-value of f-measure is 0.2179, and Friedman chi-squared is 8.2857. They all have the same degree of freedom that is 6. As a result, since the p-value of precision and recall is less than 0.05, it was statistically verified that there is a difference in performance of the algorithm. Nemenyi test has characteristics similar to Friedman test since it is usually conducted after Friedman test. It compare statistical significant between two pairwise data. By Friedman test we found that there was a difference in the defect prediction performance of the algorithm. Therefore, we conduct Nemenyi test which calculate two algorithms difference in performance. P-value is created by comparing seven classifiers with other classifiers other than oneself.

% Chapter 
\clearpage
\chapter{Discussions}
\thispagestyle{fancy}
\bigskip
In this chapter, we will discuss how this approach is important in software engineering practice, especially in automated quality assurance techniques and future work.

\section{Importance of this Approach}
Our defect-prediction approach is novel in that we built a large AED model of 196 projects to encode various change data and use the distance values achieved from the kNN model to make the prediction.
Our {\simfinmo} is fully universal which can be fully utilized in a CPDP setting without the need for historical metrics.
With this approach, we can say that the cold-start problem is fully alleviated.
Another major problem of defect prediction is the lack of actionable messages.
Traditional defect-prediction models predict the risk of a module in a file or at a method level.
For larger projects, developers need a significant amount of time to find the bug inside the risky module and fix it.
JIT-DP was actively studied to resolve the issue because code changes were typically significantly smaller than an entire file or method.
By reducing the granularity, developers can easily identify the location of the defect as the code to be inspected is shorter.
However, it does not resolve the fundamental problem.
JIT-DP does not inform the developers how the defect can be corrected.
Using {\simfinmo}, we can identify a risky change and also show the original BFC that is associated with the BIC with the nearest distance value.
The suggested BFC can act as an actionable message and provide insights to assist developers in fixing the risky change.
Fig. \ref{fig:ps_1} shows an example of a BFC that we collected from project tez, one of the testing projects.
Fig. \ref{fig:ps_2} shows an example of a suggested BFC generated by running SimFinMo on the BIC that is shown in Fig. \ref{fig:ps_1}.
Fig. \ref{fig:ps_2} shows that one parameter for calling a super constructor is added for a change.
By looking at Fig. \ref{fig:ps_1}, we can see that first and the third change is very similar to the suggested patch in that they are both adding a parameter in a method invocation.
Fig. \ref{fig:ps_2} added an AppContext instance as a parameter as its method creates a common container launcher context while Fig. 3 added an EventType info parameter as it is related to a handler method.
Based on the context of a change, developers can get an actionable message such as adding a missing parameter.
As the example illustrates, we hope that these suggested patches will help developers to act upon for quality assurance activity.
Our method of defect prediction is a novel technique that does not use a machine learning algorithm in the prediction phase.
In existing studies such as traditional, cross-project, heterogeneous, and just-in-time defect prediction, major issues are caused by the shortcomings of machine learning techniques.
Cold-start problem is caused because training data is necessary for the model to be built from the target project.
This issue is alleviated through the study of CPDP, however, CPDP does not perfectly solve the cold-start problem because many of the approaches use historical metrics as a feature. 
Because {\simfinmo} exploits existing software projects without the need for historical metrics, it fully resolves the cold-start problem.


\section{Future Work}
It is necessary to find the right threshold value to separate clean instances from buggy instances in the future.
We used different threshold values for different test projects with which we could only linearly separate the relationship among the classes.
However, more studies should be conducted to distinguish the relationship using a more complex and non-linear approach.
It was difficult to determine a global threshold that was suitable for all the projects.

Additionally, the suggested BFCs can be used as a code ingredient or mutation operator in automatic program repair (APR).
In APR, reducing the search space is critical to find the right patch within a feasible time. Ingredients or a list of operators retrieved from the suggested BFC can help reduce the search space to find the correct patch.
So these patch suggestions not only have a high potential for actionable defect prediction but also for more efficient APR technique as well.

\begin{figure}[!tbp]
\renewcommand{\arraystretch}{1}
    \centering
    \includegraphics[width=\linewidth]{figures/ps1.png}\hfill
    \caption{An example of a ground truth BFC from project tez.}
    \label{fig:ps_1}
\end{figure}

\begin{figure}[!tbp]
\renewcommand{\arraystretch}{1}
    \centering
    \includegraphics[width=\linewidth]{figures/ps2.png}\hfill
    \caption{An example of suggested BFC from the project hbase.}
    \label{fig:ps_2}
\end{figure}

% Chapter 
\clearpage
\chapter{Threats to Validity}
\thispagestyle{fancy}
\bigskip
\section{Construct Validity}
In collecting the change vectors, we have disregarded updates and move operations from the changes due to the occurrence of simple refactoring changes with no actual behavioral changes.
However, this may lead to missing some important changes such as changing the method invocation to complete another method invocation or moving an if statement to the next line to change the control flow of the program.
We believe that this sacrificed some of the true positives to eliminate many of the false positives.
For future work, handling precise updates and move conditions will enhance the change collecting accuracy and therefore enhance the ability of {\simfinmo}.

\section{External Validity}
Although we have used six of ASF projects to evaluate our {\simfinmo}, it might not represent all the projects that are in software repositories.
However, the projects are varied in their size, domain, and development time.
So we believe the results of our study have empirical value in the software engineering society.

\section{Internal Validity}
For collecting BIC data we have used the SZZ algorithm \cite{sliwerski2005changes}.
Although the algorithm is often used to collect BIC instances, it has limitations.
The deleted lines from the BFC that are blamed to trace BIC are not always bug inducing as they could be refactoring or cosmetic changes \cite{neto2018impact}.
However, with the circumstances, {\simfinmo} showed the comparable or better performance to the existing baselines.
So we believe that the better algorithm that has better precision in BIC instance collecting will improve the performance of {\simfinmo}.


% Chapter 
\clearpage
\chapter{Conclusion}
\thispagestyle{fancy}
\bigskip
We proposed a novel universal and actionable JIT-DP model that overcomes the main limitations of current defect prediction models, the lack of actionable messages, and the cold-start problem.
This study is the first of its kind in that it predicts defective modules at a change level without a machine-learning predictor. 
Although {\simfinmo} is not perfect, it does surpass some of the most used machine-learning algorithms.
Further studies in this direction will contribute to automated quality assurance techniques such as defect prediction and APR.
Studies on topics such as finding a better solution to determine a global cut-off value or collecting a higher quality of defects and clean changes will hopefully enhance the prediction performance and contribute to the provision of better actionable messages.

% References Chapter
\singlespacing
\thispagestyle{fancy}
\chapter{References}
\printbibliography[heading=subbibliography]


% \section{Variational Auto-Encoder}
% Variational Auto-Encoder (VAE) \cite{vae-kingma} is the most commonly used deep generative model based on auto-encoder model. VAE model also consists of encoder and decoder. Encoder encodes input data into latent variable which has much less dimension than input data and decoder reconstructs input data from the latent variable encoded by encoder. The difference between auto-encoder and VAE is that VAE is making a latent variable by sampling from Gaussian distribution and has a constraint on the latent space which is forced to be isotropic Gaussian by minimizing the Kullback-Leibler (KL) divergence between the Gaussian prior and the model distribution. The following Eq. \ref{eq:vae} describes the objective function of VAE. It has two terms: reconstruction error between input data and generated data, and KL divergence between the latent distribution and the Gaussian prior distribution. 
% \begin{equation}
% \mathcal{L}_{VAE} = \mathbb{E}_{q(z|x)}[\log p(x|z)] - KL(q(z|x)||p(z)),
% \label{eq:vae}
% \end{equation}
% where $KL(q||p)$ means the KL divergence between $q$ and $p$, and $q(z|x)$, $p(x|z)$ and $p(z)$ are the encoder, decoder and the prior distribution, respectively. The encoder and decoder are implemented by deep neural networks, and the prior distribution is isotropic Gaussian. See \cite{vae-kingma} for the details.

% Since the latent space generates samples for the decoder, the reparameterization trick is applied to make the gradient information flow through the latent space. After training the model, the latent space keeps most information to reconstruct input data, as well as it becomes Gaussian as much as possible. In other words, VAE cannot have perfectly disentangled latent variable even with the isotropic Gaussian prior. Fig. \ref{fig:vae-model} shows the basic model architecture of VAE.
% \begin{figure}[ht!]
% \label{fig:vae-model}
% \centerline{\hbox{
% \includegraphics[width=2.75in]{VAE_model.png}
% }} 
% \caption[VAE model]{Model architecture of VAE.}
% \end{figure}

% \section{Generative Adversarial Network}
% Generative Adversarial Network (GAN) has become one of the most famous model since it was proposed. The GAN architecture has a discriminator network and a generator network. The basic algorithm of GAN is that the discriminator is trained to distinguish the real data and generated data with latent random variable ($z$) from generator, while the generator is trained to generate data that is hard to be distinguished by the discriminator. This process is a kind of min-max game between the discriminator and the generator. The objective function of GAN can be written as Eq. \ref{eq:gan}. 
% \begin{equation}
%     \min_{G}\max_{D}V(D,G)=\mathbb{E}_{x\sim p_{data}(x)}[\log D(x)] + \mathbb{E}_{z\sim p_z(z)}[\log (1-D(G(z)))],
% \label{eq:gan}
% \end{equation}
% where $z$, $D(x)$ and $G(z)$ represent latent variable, discriminator and generator, respectively.

% The basic architecture of GAN model is presented in Fig. \ref{fig:gan-model}. Image generation models based on GAN showed remarkable results. After the basic GAN was proposed, many improved GAN models like DCGAN, WGAN, etc. have been introduced to conduct various generation tasks (e.g., style transfer, learning generating factors and etc.) and to make the training of GAN more stable. Contrary to VAE, in the conventional GAN models, there is no constraint on the latent space.

% \begin{figure}[ht!]
% \label{fig:gan-model}
% \centerline{\hbox{
% \includegraphics[width=3.75in]{GAN_model.png}
% }}
% \caption[GAN model]{Model architecture of GAN.}
% \end{figure}