This chapter explains the preliminary studies that were conducted to find different change representations.
Deciding how to represent changes is very important because the information that can be carried in a representation is very different concerning their data forms.

\section{Count-based Change Vector}
Count-based change vector representation of change has a fixed size of the sparse vector.
Each index represents a change type of an AST node.
If a certain AST node is added, the index that corresponds to the kind of change is incremented.
We capture the changed AST nodes in 4 change types, i.e. addition, deletion, updates, moves, by using a fine-grained source code differencing algorithm, Gumtree \cite{falleri2014fine}.
Because the change is represented in a numeric vector, it is suitable to be applied on numerous distance-calculating equations.
Specifically, we used the Jaccard index to find similar changes \cite{cha2007comprehensive}.

Fig. \ref{fig:cbcv_3} shows an example of two count-based change vectors that have the Jaccard index of 0.92.
Fig. \ref{fig:cbcv_1} and \ref{fig:cbcv_2} show an example of the corresponding changes.  
As depicted in the figures, we can see that both changes are similar in terms of their AST structures.
The underlined values 146 1 and 159 1 in Fig. \ref{fig:cbcv_3} means that both changes have one importDeclaration (146) and MethodInvocation (159) node added.

However, there are limitations to this representation.
Because the changes are composed of changed AST node counts, it is not able to capture the order of the changed nodes.
So changes that had very similar vectors also resulted in totally different looking changes because of different occurrences of nodes.
So to capture a more accurate representation of change, we needed a different design. 

\section{String Comparison}
To capture the order information of changes, we experimented on using the change code as a string value and used string comparing algorithms in finding similar changes.

First, we used the traditional line differencing algorithm used in git.
We didn't use Gumtree this time because Gumtree only considered the changed node without its context.
By retrieving the whole line, we could represent the change with order and context.

To reduce noise information, we only retrieved changes that both deleted lines (buggy) and added lines (patch) within the same method scope.
This way we can be sure that the change is a bug-fixing change. 
Other changes that only have either added or deleted lines are not considered.


\begin{figure}[!ht]
    \renewcommand{\arraystretch}{1}
    \centering
    \includegraphics[width=\linewidth]{figures/cbcv-1.pdf}\hfill
    \caption{A change instance in project isis.}
    \label{fig:cbcv_1}
\end{figure}
    
\begin{figure}[!ht]
    \renewcommand{\arraystretch}{1}
    \centering
    \includegraphics[width=\linewidth]{figures/cbcv-2.pdf}\hfill
    \caption{A change instance in project ignite.}
    \label{fig:cbcv_2}
\end{figure}

\begin{figure}[!ht]
    \renewcommand{\arraystretch}{1}
    \centering
    \includegraphics[width=\linewidth]{figures/cbcv-3.pdf}\hfill
    \caption{Count-based change vector if Fig. \ref{fig:cbcv_1} and \ref{fig:cbcv_2}}
    \label{fig:cbcv_3}
\end{figure}


Comparing string values take much more time than applying them to numeric values. 
So we decided to apply the SimHash algorithm to reduce the dimension and skipped comparing if they were over a certain Hamming distance \cite{uddin2011effectiveness}.
This method significantly reduced the time of similarity comparison.
We also used a mix of Jaccard index and Levenshtein distance to capture both ordering and the collective occurrence of string values. 

Fig. \ref{fig:string_1}-\ref{fig:string_4} shows an example that had a Jaccard index of 0.86. 
As depicted in the figures, changes that had the highest similarity values were usually very short. 
Some changes were longer, but they had a low similarity value.
This result seemed natural because when string values are longer there are more chances that they will be different.
So we decided on using a different method which is the {\simfin} to find similar commits in the software repositories.

For the count-based change vector and string comparison, we should find hand-designed features such as the number of AST node occurrences or string similarities.
Also, the features are limited to syntactic characteristics of code changes.
Therefore, we propose a method to find similar commits by applying deep learning so that the computer can directly derive semantic features.

\begin{figure}[!ht]
    \renewcommand{\arraystretch}{1}
    \centering
    \includegraphics[width=0.8\linewidth]{figures/string_1.pdf}\hfill
    \caption{A change instance in project lucene-solr (deleted line)}
    \label{fig:string_1}
\end{figure}

\begin{figure}[!ht]
    \renewcommand{\arraystretch}{1}
    \centering
    \includegraphics[width=0.8\linewidth]{figures/string_2.pdf}\hfill
    \caption{A change in project lucene-solr (added line)}
    \label{fig:string_2}
\end{figure}

\begin{figure}[!ht]
    \renewcommand{\arraystretch}{1}
    \centering
    \includegraphics[width=0.8\linewidth]{figures/string_3.pdf}\hfill
    \caption{A change instance in project oozie (deleted line)}
    \label{fig:string_3}
\end{figure}

\begin{figure}[!ht]
    \renewcommand{\arraystretch}{1}
    \centering
    \includegraphics[width=0.8\linewidth]{figures/string_4.pdf}\hfill
    \caption{A change instance in project oozie (added line)}
    \label{fig:string_4}
\end{figure}